\documentclass[15pt,a4paper]{article}
\usepackage[utf8]{inputenc}
\usepackage{amsmath}
\usepackage{amsfonts}
\usepackage{amssymb}
\usepackage{graphicx}


\begin{document}
\title{Mathematics}
\author{Mamadou Alieu Jallow}
\date{\today}
\maketitle
% puting the table of contents 
\tableofcontents
\section{Fraction}
\subsection{Type of Fraction}
\subsection{Problem solving }
\subsection{Word Problem}

\section{Sets}
\ Set can be defined as the collection of objects or items .In dealing with set there is a very important rule that we have to know that is there is no repeating of elements in the listing of the elements, There are different ways of representating sets

\subsection{Realatioonship betwween sets}
\begin{enumerate}
\item Universal Set: This is the set that contain all the elements in the set and it is usually denoted by the symbol$\cup $.

\item Union : The union of two set A and B is written as 
$A\cup B$. it simplies writtng all the elements in A and B without repeating any elements
% will have to check it again and make the intersection big with the character 
\item Intersection : This intersection of two set A and B is written as 
$A\bigcap B$
\item Empty set: This is a set with no elements in it .it is usually denoted by 
$\lbrace\rbrace$ or $\phi$
\item Complement : The complement of a set A are those elements that are in the universal set but not in the set A .It is usually denoted by  $ A^{c}$ or$ A^{1}$ 
\end{enumerate}
\subsection{problem solving }
\ The Following are examples that are
\begin{enumerate}
 \item  $A=\lbrace 1,2 ,5,7\rbrace and  B=\lbrace1,3,6,7\rbrace $ are subset of the universal ser $\cup =\lbrace1,2,3\cdots,10\rbrace$.Find 
 \begin{enumerate}
 \item  $  A^{c}$
 \item $\langle A\cap B\rangle^{c}$
 \item $\langle A\cup B\rangle^{c} $
 \item The subset of B each of which has three elements 
% this is  the first problem 
 \end{enumerate}
 \item The set of $ A=\lbrace1,3,5,7,9,11\rbrace ,B=\lbrace 2,3,5,7,11,15\rbrace and C=\lbrace 3,6,9,15\rbrace $ are subset of $\xi=\lbrace1,2,3,...\cdots 15\rbrace $.
 \begin{enumerate}
 \item Draw a venn diagram to illustrate the given information 
 \item Use your diagram to find the 
 \begin{enumerate}
  \item $ C\cap A $
 \end{enumerate}
  \end{enumerate}
 \end{enumerate}
\subsection{word pdroblem}
\begin{enumerate}
\item In a team of 30 girls ,18 play netball and 14 play hockey .if 5 play neither .find the number of girls that play both .
\item in a examination, 60 candidates offered maths ,80 offered english  and 50 offered physics .if 20 offered maths and english .if offered english and physics .25 offered maths and physics and 10 offered all the three subjects .how many students sat fot the examination .
   
\end{enumerate}

\section{Modulo arithmetic}
\subsection{opeartion in modulo}
\subsection{Equation in modulo}
\ The following are opertion that are carried out in an modular arithmetic
\subsection{problem solving }
\end{document}